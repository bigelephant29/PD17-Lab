\documentclass[t]{beamer}

\usetheme{CambridgeUS}

\title{IM1003: Programming Design, Spring 2017  \linebreak Lab 08}
\author[bigelephant29]{Jhih-Bang Hsieh\linebreak \small{bigelephant29}}
\institute{\textbf{National Taiwan University}}
\date{}

\usefonttheme{serif}
\usepackage{xeCJK} 
\usepackage{fontspec}
\setCJKmainfont{DFPHeiMedium-B5}

\usepackage{graphicx}
\graphicspath{{/}}

\usepackage{listings}
\usepackage{color}

\definecolor{dkgreen}{rgb}{0,0.6,0}
\definecolor{gray}{rgb}{0.5,0.5,0.5}
\definecolor{mauve}{rgb}{0.58,0,0.82}

\lstset{
  language=C++,
  showstringspaces=false,
  columns=flexible,
  basicstyle={\linespread{0.6}\scriptsize\ttfamily},
  numbers=left,
  firstnumber=1,
  numberfirstline=true,
  numberstyle=\tiny\color{gray},
  keywordstyle=\color{blue},
  commentstyle=\color{dkgreen},
  stringstyle=\color{mauve},
  breaklines=true,
  breakatwhitespace=true,
  tabsize=4,
  xleftmargin=4em
}

\makeatletter
\setbeamertemplate{footline}{%
    \leavevmode%
    \hbox{%
        \begin{beamercolorbox}[wd=.8\paperwidth,ht=2.25ex,dp=1ex,center]{title in head/foot}%
            \usebeamerfont{title in head/foot}\insertshorttitle
        \end{beamercolorbox}%
        \begin{beamercolorbox}[wd=.2\paperwidth,ht=2.25ex,dp=1ex,right]{date in head/foot}%
            \usebeamerfont{date in head/foot}\insertshortdate{}\hspace*{2em}
            \insertframenumber{} / \inserttotalframenumber\hspace*{2ex} 
        \end{beamercolorbox}}%
        \vskip0pt%
    }
\makeatother

\begin{document}

% First Page
\begin{frame}
  \maketitle
\end{frame}

% Outline Page
\begin{frame}{Outline}
  \begin{itemize}
    \item Bitwise operation
    \item Practice
  \end{itemize}
\end{frame}

\section{Bitwise Operation}
\begin{frame}{Bitwise Operation}
  \begin{itemize}
    \item 在電腦的世界裡,所有的數字都是二進位。
    \item 對電腦來說,如果我們可以直接用二進位運算,效率很高。
    \item $(1234)_{10} = (10011010010)_{2}$
    \item 常用的 Operator:
      \begin{itemize}
        \item Left Shift $<<$
        \item Right Shift $>>$
        \item AND $\&$
        \item OR $|$
        \item XOR $\textasciicircum$
        \item NOT $\sim$
      \end{itemize}
  \end{itemize}
\end{frame}

\subsection{Left Shift}
\begin{frame}{Left Shift $<<$}
  \begin{itemize}
    \item 在二進位下,將某數加上 k 位。
    \item $1234\ <<\ 2 = 4936$
    \item $(1234)_{10} = (10011010010)_{2}$
    \item $(4936)_{10} = (1001101001000)_{2}$
    \item 相當於某數乘上 $2^{k}$。
    \item 高位溢位自動消失。
  \end{itemize}
\end{frame}

\subsection{Right Shift}
\begin{frame}{Right Shift $>>$}
  \begin{itemize}
    \item 在二進位下,將某數去掉末 k 位。
    \item $1234\ >>\ 2 = 308$
    \item $(1234)_{10} = (10011010010)_{2}$
    \item $(308)_{10} = (100110100)_{2}$
    \item 相當於某數除以 $2^{k}$。
    \item 低位自動消失。
  \end{itemize}
\end{frame}

\subsection{AND}
\begin{frame}{AND $\&$}
  \begin{itemize}
    \item 在二進位下,取兩者皆 1 則 1,否則為 0。
    \item $1234\ \&\ 147 = 146$
    \item $(1234)_{10} = (10011010010)_{2}$
    \item $(147)_{10} = (10010011)_{2}$
    \item $(146)_{10} = (10010010)_{2}$
  \end{itemize}
\end{frame}

\subsection{OR}
\begin{frame}{OR $|$}
  \begin{itemize}
    \item 在二進位下,取兩者皆 0 則 0,否則為 1。
    \item $1234\ |\ 147 = 1235$
    \item $(1234)_{10} = (10011010010)_{2}$
    \item $(147)_{10} = (10010011)_{2}$
    \item $(1235)_{10} = (10011010011)_{2}$
  \end{itemize}
\end{frame}

\subsection{XOR}
\begin{frame}{XOR $\textasciicircum$}
  \begin{itemize}
    \item 在二進位下,取兩者相異為 1,否則為 0。
    \item $1234\ \textasciicircum\ 147 = 1089$
    \item $(1234)_{10} = (10011010010)_{2}$
    \item $(147)_{10} = (10010011)_{2}$
    \item $(1089)_{10} = (10001000001)_{2}$
  \end{itemize}
\end{frame}

\subsection{NOT}
\begin{frame}{NOT $\sim$}
  \begin{itemize}
    \item 在二進位下,0 變 1,1 變 0。
    \item $\sim 1234 = -1235$
    \item $(1234)_{10} = (10011010010)_{2}$
    \item $(-1235)_{10} = (11111111111111111111101100101101)_{2}$
  \end{itemize}
\end{frame}

\subsection{技巧}
\begin{frame}{交換兩數}
  \begin{itemize}
    \item a\ \textasciicircum=\ b\ \textasciicircum=\ a\ \textasciicircum=\ b
    \item 這樣其實很慢。
  \end{itemize}
\end{frame}

\begin{frame}{奇偶}
  \begin{itemize}
    \item n\ \&\ 1
    \item 意思跟 n \% 2 一樣
    \item 注意 \& 運算的優先序,很容易出事
  \end{itemize}
\end{frame}

\section{Practice}
\begin{frame}{*Practice K - Lowbit}
  給一個正整數 $N$,輸出 $N$ 在二進位下的最小位數(以 2 的冪次輸出)。\\
  例如:$72$ 轉成二進位是 $1001000$,故輸出 $8$($1000$)。
\end{frame}

\begin{frame}{Practice L}
  給一個長度為 $N$ 的數列,輸出整個數列 XOR 的值。\\
  範例輸入:\\
  $5$\\
  $1\ 2\ 3\ 5\ 8$\\
  範例輸出:\\
  $13$
\end{frame}

\begin{frame}{*Practice M - Switches}
  現在有很多的開關,開關只有開與關兩種狀態。\\
  給一個長度為 $N$ 數列,數列中的每一項代表將一個開關轉換狀態(開變關、關變開)。\\
  保證數列長度是奇數,並且最後只剩下一個開著的開關。\\
  請輸出剩下開著的開關編號。
\end{frame}

\end{document}
