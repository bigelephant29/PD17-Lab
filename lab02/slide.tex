\documentclass[t]{beamer}

\usetheme{CambridgeUS}

\title{IM1003: Programming Design, Spring 2017  \linebreak Lab 02}
\author[bigelephant29]{Jhih-Bang Hsieh\linebreak \small{bigelephant29}}
\institute{\textbf{National Taiwan University}}
\date{}

\usefonttheme{serif}
\usepackage{xeCJK} 
\usepackage{fontspec}
\setCJKmainfont{DFPHeiMedium-B5}

\usepackage{graphicx}
\graphicspath{{/}}

\usepackage{listings}
\usepackage{color}

\definecolor{dkgreen}{rgb}{0,0.6,0}
\definecolor{gray}{rgb}{0.5,0.5,0.5}
\definecolor{mauve}{rgb}{0.58,0,0.82}

\lstset{
  language=C++,
  showstringspaces=false,
  columns=flexible,
  basicstyle={\scriptsize\ttfamily},
  numbers=left,
  firstnumber=1,
  numberfirstline=true,
  numberstyle=\tiny\color{gray},
  keywordstyle=\color{blue},
  commentstyle=\color{dkgreen},
  stringstyle=\color{mauve},
  breaklines=true,
  breakatwhitespace=true,
  tabsize=4,
  xleftmargin=4em
}

\makeatletter
\setbeamertemplate{footline}{%
    \leavevmode%
    \hbox{%
        \begin{beamercolorbox}[wd=.8\paperwidth,ht=2.25ex,dp=1ex,center]{title in head/foot}%
            \usebeamerfont{title in head/foot}\insertshorttitle
        \end{beamercolorbox}%
        \begin{beamercolorbox}[wd=.2\paperwidth,ht=2.25ex,dp=1ex,right]{date in head/foot}%
            \usebeamerfont{date in head/foot}\insertshortdate{}\hspace*{2em}
            \insertframenumber{} / \inserttotalframenumber\hspace*{2ex} 
        \end{beamercolorbox}}%
        \vskip0pt%
    }
\makeatother

\begin{document}

% First Page
\begin{frame}
  \maketitle
\end{frame}

% Outline Page
\begin{frame}{Outline}
  \begin{itemize}
    \item if-else
    \item Logical Operators
    \item switch-case
    \item while, do-while
    \item for
    \item Nested Structure
    \item Practice
  \end{itemize}
\end{frame}

% Section: if-else 
\section{if-else}
\begin{frame}{if-else}
  \begin{itemize}
    \item 三個組成元件:if、else if、else
    \item if 為一切的開頭
    \item else if 只能接在 if 或 else if 後面
    \item else 只能接在 if 或 else if 後面
    \item 一旦寫了 else,後面就不能再接任何條件
  \end{itemize}
  \begin{itemize}
    \item 如果今天是晴天,帥哥哥會想出去運動。(if)
    \item 否則如果今天是陰天,帥哥哥會想出去逛街。(else if)
    \item 否則的話,帥哥哥會想宅在家裡打 LOL。(else)
  \end{itemize}
\end{frame}

\begin{frame}{if-else}
  \begin{itemize}
    \item 你不會劈頭就說「否則」,同理 else if、else 不會當作條件式的開頭。
    \item 你可以一直有各種特例,所以有很多「否則如果(else if)」是很合理的。
    \item 最後一個否則,如果沒有帶有條件,則囊括所有其他例外。
    \item 囊括所有例外的否則(else)最多只有一個。
  \end{itemize}
\end{frame}

\section{Logical Operators}
\begin{frame}{Logical Operators}
  \begin{itemize}
    \item Logical and: \&\&
    \item Logical or: ||
    \item Logical not: !
    \item Unary Operator(單元運算子):!
    \item Binary Operator(二元運算子):||、!
    \item 不確定運算優先順序的話,記得加個括號!
  \end{itemize}
\end{frame}

\section{switch-case}
\begin{frame}{switch-case}
  \lstinputlisting[breaklines]{code/switch.cpp}
  \begin{itemize}
    \item case 跟 case 之間記得要 break。
    \item 多條件可以省略 break,如範例的 value\_1、value\_2。
  \end{itemize}
\end{frame}

\begin{frame}{switch-case}
  \lstinputlisting[breaklines]{code/switch_ce.cpp}
  \begin{itemize}
    \item error: redefinition of 'tmp'
    \item 因為所有 case 目前屬於同一個 block,所以這樣寫是不好的。
    \item 盡量避免在 switch-case 裡面宣告變數。
    \item 不得已的時候,請加上大括號!
  \end{itemize}
\end{frame}

\begin{frame}{switch-case}
  \lstinputlisting[breaklines]{code/switch_ok.cpp}
\end{frame}

\section{while, do-while}
\begin{frame}{while, do-while}
  \begin{itemize}
    \item do-while 非常少用到,但是有些時候它非常好用。
    \item 無論任何條件下,至少必須執行一次的 while 迴圈。
    \item 大家可以回想一下 Lab01 Practice B!
  \end{itemize}
\end{frame}

\section{for}
\begin{frame}{for}
  \begin{itemize}
    \item 帶有計數器(counter)的迴圈。
    \item 使用時必須清楚了解 for 的執行順序!
  \end{itemize}
\end{frame}

\begin{frame}{for}
  \lstinputlisting[breaklines]{code/for.cpp}
  \begin{enumerate}
    \item 執行 (1) 進行初始化,在這裡可以宣告暫時變數。
    \item 判斷 (2) 條件是否成立,成立則繼續,否則離開。
    \item 執行 (4)。
    \item 執行 (3)。
    \item 回到第 2 點。
  \end{enumerate}
\end{frame}

\begin{frame}{for}
  \lstinputlisting[breaklines]{code/for_1.cpp}
  \lstinputlisting[breaklines]{code/for_2.cpp}
\end{frame}

\section{Nested Structure}
\begin{frame}{Nested Structure}
  \begin{itemize}
    \item 剛剛介紹的所有結構,都可以寫成巢狀結構。(好潮ㄛ)
    \item 巢狀結構賦予程式碼更多彈性,可以組成更強、更簡潔的邏輯架構。
    \lstinputlisting[breaklines]{code/nest.cpp}
  \end{itemize}
\end{frame}

\section{Practice}
\begin{frame}{Practice}
  請以本週所學,實作上週的 Practice A、B、C、D。
\end{frame}

\subsection{E}
\begin{frame}{Practice E}
  請用巢狀結構輸出九九乘法表,沒有格式限制。
\end{frame}

\end{document}
