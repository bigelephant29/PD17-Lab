\documentclass[t]{beamer}

\usetheme{CambridgeUS}

\title{IM1003: Programming Design, Spring 2017  \linebreak Lab 03}
\author[bigelephant29]{Jhih-Bang Hsieh\linebreak \small{bigelephant29}}
\institute{\textbf{National Taiwan University}}
\date{}

\usefonttheme{serif}
\usepackage{xeCJK} 
\usepackage{fontspec}
\setCJKmainfont{DFPHeiMedium-B5}

\usepackage{graphicx}
\graphicspath{{/}}

\usepackage{listings}
\usepackage{color}

\definecolor{dkgreen}{rgb}{0,0.6,0}
\definecolor{gray}{rgb}{0.5,0.5,0.5}
\definecolor{mauve}{rgb}{0.58,0,0.82}

\lstset{
  language=C++,
  showstringspaces=false,
  columns=flexible,
  basicstyle={\scriptsize\ttfamily},
  numbers=left,
  firstnumber=1,
  numberfirstline=true,
  numberstyle=\tiny\color{gray},
  keywordstyle=\color{blue},
  commentstyle=\color{dkgreen},
  stringstyle=\color{mauve},
  breaklines=true,
  breakatwhitespace=true,
  tabsize=4,
  xleftmargin=4em
}

\makeatletter
\setbeamertemplate{footline}{%
    \leavevmode%
    \hbox{%
        \begin{beamercolorbox}[wd=.8\paperwidth,ht=2.25ex,dp=1ex,center]{title in head/foot}%
            \usebeamerfont{title in head/foot}\insertshorttitle
        \end{beamercolorbox}%
        \begin{beamercolorbox}[wd=.2\paperwidth,ht=2.25ex,dp=1ex,right]{date in head/foot}%
            \usebeamerfont{date in head/foot}\insertshortdate{}\hspace*{2em}
            \insertframenumber{} / \inserttotalframenumber\hspace*{2ex} 
        \end{beamercolorbox}}%
        \vskip0pt%
    }
\makeatother

\begin{document}

% First Page
\begin{frame}
  \maketitle
\end{frame}

% Outline Page
\begin{frame}{Outline}
  \begin{itemize}
    \item 測資組
    \item Testing
    \item Practice
  \end{itemize}
\end{frame}

\section{測資組}
\begin{frame}{測資組}
  \begin{itemize}
    \item 從本次作業開始,我們將開始使用測資組進行評分。
    \item 產出能夠幫所有人檢查錯誤的測資不容易,所以需要海量的測資。
    \item 為了不要讓分數太難拿,大部分的測資只有一筆。
    \item 本次作業中,佔分 $75\%$ 的測資為單筆測資,剩下配分分成 $5$ 組,每一組 $25$ 筆測資。
    \item 不含範例總共有 $15 + 5 \times 25 = 140$ 筆測試資料。
    \item 希望大家上傳前多多檢查,一不小心大家要等你 $141$ 秒 QQ。
    \item 希望大家可以提早做作業,死線前被大量超時的上傳卡住,助教不會負責。
  \end{itemize}
\end{frame}

% Test
\section{Testing}
\begin{frame}{Testing}
  \begin{itemize}
    \item 測試(Testing)是軟體工程師必須具備的基本能力之一。
    \item 我們必須根據程式的目標以及規格書(spec),對程式進行基本的測試。
    \item 我們可以追蹤程式執行過程中的變數值,以及一些必備的條件。
  \end{itemize}
\end{frame}

\begin{frame}{微波爐}
  \begin{itemize}
    \item 我們都知道微波爐不能用金屬容器微波。
    \item 用金屬容器的話會爆炸!
    \item 如果我們可以在開始微波前,對容器類型做測試,則可以避免金屬容器!
    \item 測試(容器)
  \end{itemize}
\end{frame}

\subsection{Assertion}
\begin{frame}{Assertion}
  \begin{itemize}
    \item assert(斷言)是一種程式開發過程中用來驗證程式結果是否如預期的功能。
    \item 當執行過程中,assert 的值出現 False 時,程式會被強制結束並產生錯誤訊息。
    \item 在 C++ 中,請引入 <cassert> 標頭檔。
    \item assert(值);
  \end{itemize}
\end{frame}

\subsection{Special Case}
\begin{frame}{Special Case}
  \begin{itemize}
    \item 針對不同的題目,我們可以去思考一些特殊的情況,致使自己的程式出錯
    \item 當沒有頭緒的時候,可以從輸入的條件開始思考(測試邊界條件)。
    \item 如果題目說 $1\le n\le 100$,那 $n=1$、$n=100$ 的狀況都要試試看。
  \end{itemize}
\end{frame}

\begin{frame}{Example: HW 2-3}
  \begin{itemize}
    \item 給你一個不超過 $10000$ 的數字,試列出所有因數、所有質因數。
    \item Case 1: 極大、極小的質數。
    \item Case 2: $2$、$3$ 的冪次($2^k$、$3^k$),或其組合。
    \item Case 3: $2$、$3$ 的冪次減 $1$($2^k-1$、$3^k-1$),或其組合。
  \end{itemize}
\end{frame}

\begin{frame}{Example: HW 2-4}
  \begin{itemize}
    \item 輸入一個數字,輸出其質因數分解。
    \item Case 1: 質數。
    \item Case 2: 質數的冪次($p^k$)。
    \item Case 3: 單一質因數一次方的組合。
    \item Case 4: 單一質因數多次方的組合。
    \item 還有很多 case,我們也可以把題目倒過來想!
  \end{itemize}
\end{frame}

\section{Practice}
\subsection{F}
\begin{frame}{Practice F - Rotation}
  給你一個數列。我們說對數列進行一次「Rotate」代表將該數列首項移到最後,
  產生一個長度相同的新數列。現在,我們希望你可以幫我們對數列做 Rotate。\\
  \vspace{2em}
  舉例來說:\\
  1 2 3 4 5 Rotate 3 次 會得到 4 5 1 2 3。
\end{frame}

\subsection{}
\begin{frame}{Practice}
  \begin{enumerate}
    \item 請用本週教的方法對之前的作業、練習進行測試。
    \item 之前有錯的程式碼也可以拿來練習測試喔!
    \item 請設想一些本週作業可能會出現的特殊情況。
    \item 有任何的想法,歡迎與助教討論。
  \end{enumerate}
\end{frame}

% Section: HW2 Code Review
\end{document}
