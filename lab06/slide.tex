\documentclass[t]{beamer}

\usetheme{CambridgeUS}

\title{IM1003: Programming Design, Spring 2017  \linebreak Lab 06}
\author[bigelephant29]{Jhih-Bang Hsieh\linebreak \small{bigelephant29}}
\institute{\textbf{National Taiwan University}}
\date{}

\usefonttheme{serif}
\usepackage{xeCJK} 
\usepackage{fontspec}
\setCJKmainfont{DFPHeiMedium-B5}

\usepackage{graphicx}
\graphicspath{{/}}

\usepackage{listings}
\usepackage{color}

\definecolor{dkgreen}{rgb}{0,0.6,0}
\definecolor{gray}{rgb}{0.5,0.5,0.5}
\definecolor{mauve}{rgb}{0.58,0,0.82}

\lstset{
  language=C++,
  showstringspaces=false,
  columns=flexible,
  basicstyle={\linespread{0.6}\scriptsize\ttfamily},
  numbers=left,
  firstnumber=1,
  numberfirstline=true,
  numberstyle=\tiny\color{gray},
  keywordstyle=\color{blue},
  commentstyle=\color{dkgreen},
  stringstyle=\color{mauve},
  breaklines=true,
  breakatwhitespace=true,
  tabsize=4,
  xleftmargin=4em
}

\makeatletter
\setbeamertemplate{footline}{%
    \leavevmode%
    \hbox{%
        \begin{beamercolorbox}[wd=.8\paperwidth,ht=2.25ex,dp=1ex,center]{title in head/foot}%
            \usebeamerfont{title in head/foot}\insertshorttitle
        \end{beamercolorbox}%
        \begin{beamercolorbox}[wd=.2\paperwidth,ht=2.25ex,dp=1ex,right]{date in head/foot}%
            \usebeamerfont{date in head/foot}\insertshortdate{}\hspace*{2em}
            \insertframenumber{} / \inserttotalframenumber\hspace*{2ex} 
        \end{beamercolorbox}}%
        \vskip0pt%
    }
\makeatother

\begin{document}

% First Page
\begin{frame}
  \maketitle
\end{frame}

% Outline Page
\begin{frame}{Outline}
  \begin{itemize}
    \item Introduction to competitive programming
    \item Online Programming Competitions
    \item Practice
  \end{itemize}
\end{frame}

% Section 1: Introduction to competitive programming
\section{Competitive Programming}
\subsection{ACM-ICPC}
\begin{frame}{ACM-ICPC}
  \begin{itemize}
    \item ACM-ICPC 國際大學生程序設計競賽
    \item International Collegiate Programming Contest
    \item 主要以演算法、資料結構解題為主。
    \item 形式就像大家的作業那樣。
    \item 比賽時間五小時,有氣球、點心。
    \item 國內很多大學開設「競技程式設計」課程培養選手。
    \item 臺大拿了三次金牌。
  \end{itemize}  
  \begin{center}
    \includegraphics[height=5em]{image/icpc.png}
  \end{center}
\end{frame}

\begin{frame}{ACM-ICPC 賽制}
  \begin{itemize}
    \item 時限五小時,總共有 10 到 15 題。
    \item 三人一隊,只有一台電腦。
    \item 可攜帶 25 頁單面印刷 A4 參考資料。(Codebook)
    \item 通常會跨用餐時間,所以會有大量點心!
    \item 每解一題就有一顆氣球,每道題目都有不同顏色的氣球。
    \item 首先比解題數,再比用時,最後比上傳的時間點。
    \item 所有通過的題目裡面,錯誤的嘗試每次扣 20 分鐘。
  \end{itemize}
\end{frame}

\begin{frame}{ACM-ICPC 賽制}
  \begin{itemize}
    \item 下半年(上學期)於各地舉行區域賽(regional)。
    \item 於各區域賽取得 World Final Slot 者,可由學校推派至世界大賽。
    \item 一校至多一隊,於隔年上半年舉行。
    \item 一個賽區的 Slot 大概一至三張,視報名隊伍以及國內外隊伍比例決定。
  \end{itemize}
\end{frame}

\begin{frame}{台大 ACM-ICPC 資格}
  \begin{itemize}
    \item \href{http://acm.csie.org/ntujudge/index.php}{\underline{NTU Online Judge}}
    \item 每年七月初報名,七月中考試。
    \item 完成課前作業獲得滿分者,或通過個人預賽門檻者,進入團體預賽。
    \item 一至二週組隊時間,進行團體預賽。
    \item 通過團體預賽者進入 ACM 培訓班,每次可領三學分,至多領六學分。
    \item 從八月開始一直到十二月,每週六練習五小時。
    \item 每週有難到炸開的作業十題。
    \item 團體排名賽中獲得前幾名的隊伍可以拿補助出國征戰區域賽!
  \end{itemize}
\end{frame}

\begin{frame}{台大 ACM-ICPC 資格}
  \begin{itemize}
    \item 台大會去的賽區:台灣、日本、南韓、越南、泰國、印尼、新加坡、孟加拉。
    \item 有些賽區不會每年都有。
  \end{itemize}
\end{frame}

\subsection{大甲賽}
\begin{frame}{大甲賽}
  \begin{itemize}
    \item 大甲 = 大專甲組
    \item 需要額外選拔校隊的比賽。
    \item 每年十月左右選拔,十一月左右出隊。
    \item 整個 ACM 培訓班都會參加預賽。
    \item 比賽結果會影響到 ACM-ICPC 國內賽區的出隊數量。
    \item 預賽可「免費」「自由報名」,可以吃好吃的餐盒。
    \item 可以去嚐嚐被培訓班揍歪的感覺,台大校隊很強!
  \end{itemize}
\end{frame}

\subsection{PDAO}
\begin{frame}{NTUIM PDAO}
  \begin{itemize}
    \item 資管系的比賽!
    \item 兩年前第一次舉辦,時間在五月中。
    \item 有點心、有氣球,而且獎品還不錯!
    \item 聽說今年要辦(?)不過還沒確定
    \item 第一屆的考古題在 PDOGS 上面看得到喔!
  \end{itemize}
\end{frame}

\subsection{how to practice}
\begin{frame}{練習資源}
  \begin{itemize}
    \item AtCoder
    \item CodeForces
    \item LeetCode
  \end{itemize}
\end{frame}

\begin{frame}{AtCoder}
  \begin{itemize}
    \item url: \href{https://atcoder.jp/}{\underline{AtCoder}}
    \item 日本的 Judge,去年暑假才開放給外國人使用。
    \item 過去會幫日本國內舉辦比賽。
    \item 幾乎週週有比賽,有新手向的分區。
    \item 比賽時間很友善,大部份是台灣時間晚上八點到十點。
    \item 概念就跟 LOL 差不多,可以爬分。(還會分段位)
    \item 所有人的程式碼都公開,可以觀摩別人的寫法。
  \end{itemize}
\end{frame}

\begin{frame}{CodeForces}
  \begin{itemize}
    \item url: \href{http://codeforces.com/}{\underline{CodeForces}}
    \item 俄羅斯的 Judge,主流 Judge 中歷史悠久。
    \item 比賽時間、日期都比較不固定,但會主動寄信通知。
    \item 比賽時間大多不友善,通常半夜十二點半開始。
    \item 一樣可以爬分,而且比 AtCoder 難爬。
    \item 偶爾會有徵才賽、贊助商賽等,蠻有趣的。
    \item 所有人程式碼一樣公開!
  \end{itemize}
\end{frame}

\begin{frame}{LeetCode}
  \begin{itemize}
    \item url: \href{https://leetcode.com/}{\underline{LeetCode}}
    \item 所有題目都是各大公司的面試題。
    \item 每道題目都會要求你寫一個 Function。
    \item 題目假設比較模糊一點(畢竟面試也是這樣)。
    \item 以練習來說,不太算是一個很好的平台。
    \item 模擬面試實戰經驗倒是一個不錯的選擇!
  \end{itemize}
\end{frame}

\section{Online Programming Contest}
\begin{frame}{Online Programming Contest}
  \begin{itemize}
    \item Facebook Hacker Cup
    \item Google Code Jam 與它的快樂好夥伴們
  \end{itemize}
\end{frame}

\subsection{Facebook Hacker Cup}
\begin{frame}{Facebook Hacker Cup}
  \begin{itemize}
    \item url: \href{https://www.facebook.com/hackercup/}{\underline{Facebook Hacker Cup}}
    \item Facebook 辦的比賽,在一月中。
    \item 2017 賽制技能樹:
      \begin{itemize}
        \item 資格賽(答對至少一題晉級)
        \item Round 1(答對至少 35 分晉級)
        \item Round 2(前兩百名晉級,前五百名拿到紀念 T-shirt)
        \item Round 3(前 25 名晉級決賽)
        \item World Final
      \end{itemize}
    \item 資格賽進行三天,Round 1 進行一天,Round 2 開始限時三小時。
    \item 每道題目只有一次機會,啟動機會後會獲得測試資料。
    \item 啟動後六分鐘內不限次數上傳,以最後一次上傳計分。
    \item 很容易爆炸的比賽,而且題目很難。
  \end{itemize}
\end{frame}

\subsection{Google Code Jam}
\begin{frame}{Google Code Jam}
  \begin{itemize}
    \item url: \href{https://code.google.com/codejam/}{\underline{Google Code Jam}}
    \item Google 的一個團隊,最大的比賽就是 Code Jam。
    \item Code Jam 比賽時間從四月一路比到六月,總決賽在七、八月。
    \item 2017 賽制技能樹:
      \begin{itemize}
        \item 資格賽(答對指定分數晉級,比賽開始公告分數)
        \item Round 1A、Round 1B、Round 1C(前一千名晉級)
        \item Round 2(前五百名晉級,前一千名拿到紀念 T-shirt)
        \item Round 3(前 25 名晉級)
        \item World Final
      \end{itemize}
    \item 資格賽進行 27 小時,Round 1 開始限時三小時。
    \item 大部分題目分成 Small / Large 兩種 case。
    \item 通過每一道題目的 Small 才可以挑戰 Large。
    \item 開始挑戰後八分鐘內不限次數上傳,以最後一次上傳計分。
    \item Small 不限挑戰次數,錯誤的挑戰在答對該題後會轉成罰分。
  \end{itemize}
\end{frame}

\subsection{Practice}
\begin{frame}{Google Code Jam 2014 Qualification Round}
  url: \href{https://code.google.com/codejam/contest/2974486/dashboard}{\underline{GCJ 2014 Qualification}}
\end{frame}

\end{document}
